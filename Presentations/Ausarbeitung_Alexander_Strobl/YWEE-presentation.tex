\documentclass[12pt]{beamer}
\let\Tiny=\tiny
\usetheme{Air}
\usepackage{wasysym}
\usepackage{ucs}
\usepackage[utf8x]{inputenc}
\usepackage[ngerman]{babel}
\usepackage{verbatim}
\usepackage{listings}
\usepackage{color}
\usepackage{textcomp}

%%%%%%%%%%%%%%%%%%% JAVA SCRIPT SUPPORT %%%%%%%%%%%%%%%%%%%%%%%%
\definecolor{lightgray}{rgb}{.9, .9, .9}
\definecolor{darkgray}{rgb}{.4, .4, .4}
\definecolor{purple}{rgb}{0.65, 0.12, 0.82}

\lstdefinelanguage{JavaScript}{
        keywords={typeof, new, true, false, catch, function, return, null, catch, switch, var, if, in, while, do, else, case, break},
        keywordstyle=\color{blue}\bfseries,
        ndkeywords={class, export, boolean, throw, implements, import, this},
        ndkeywordstyle=\color{darkgray}\bfseries,
        identifierstyle=\color{black},
        sensitive=false,
        comment=[l]{//},
        morecomment=[s]{/*}{*/},
        commentstyle=\color{purple}\ttfamily,
        stringstyle=\color{red}\ttfamily,
        morestring=[b]',
        morestring=[b]"
        }

\pdfinfo{
  /Title       (YWEE --- Personal Tutoring Service)
  /Creator     (TeX)
  /Author      (Alexander Strobl --- Gruppe 2)
}

\title{YWEE}
\subtitle{Personal Tutoring Service \\ Gruppe 2}
\author{Alexander Strobl}
\date{02. Juli 2014}

\begin{document}

\frame{\titlepage}

\section*{}
\begin{frame}
  \frametitle{Agenda}
  \tableofcontents[hidesubsections]
\end{frame}

\AtBeginSection[]
{
  \frame<handout:0>
  {
    \frametitle{Agenda}
    \tableofcontents[currentsection, currentsubsection, sectionstyle=show/shaded, subsectionstyle=show/show/hide, subsubsectionstyle=show/hide]
  }
}


\newcommand<>{\highlighton}[1]{%
  \alt#2{\structure{#1}}{{#1}}
}

\newcommand{\icon}[1]{\pgfimage[height=1em]{#1}}

%%%%%%%%%%%%%%%%%%%%%%%%%%%%%%%%%%%%%%%%%
%%%%%%%%%% YWEE content starts here %%%%%%%%%%
%%%%%%%%%%%%%%%%%%%%%%%%%%%%%%%%%%%%%%%%%
% Einleitung
\section{Einleitung} % "Sowas wie Kapitel"
\subsection{Die Aufgabe} % Unter Kapitel

\begin{frame} %%Eine Folie
%   \frametitle{Tutoren Agentur} %%Folientitel
% Definitionsblock
	\begin{block}{\textbf{Projektdefinition}}
	  {\small Designen und Implementieren einer webbasierten Anwendung um Personen zu helfen einen Tutor zu finden, sowie dass anbieten der Dienste von Tutoren}
	\end{block}
\end{frame}


% SRA
\section{SRA}
\subsection{Die Vorgabe}
\begin{frame} %%Eine Folie
  \frametitle{Karte} %%Folientitel

Implementierung einer Karte mit folgenden Vorgaben:
%   Auflsitung mit Punkt davor
  \begin{itemize}
   \item Standartkarte ohne Routenberechnung
   \item Berechnung einer Route, wobei der Ausgangspunkt dynamisch ist
   \item Speicherung der Standortdaten
  \end{itemize}
\end{frame}

\begin{frame} %%Eine Folie
  \frametitle{Video} %%Folientitel 
Implementierung eines Videos mit folgenden Vorgaben:
%   Auflsitung mit Punkt davor
  \begin{itemize}
   \item Kontrollfunktion
   \item Browserkompatibilität
  \end{itemize}
\end{frame}
 
\begin{frame} %%Eine Folie
  \frametitle{Slideshow} %%Folientitel
Implementierung einer Slideshow mit folgenden Vorgaben:
%   Auflsitung mit Punkt davor
  \begin{itemize}
   \item Abspielen ohne JavaScript möglich
   \item Soll Breite des gesamten Headers ausfüllen
  \end{itemize}
\end{frame}


\subsection{Die Formulierung}
\begin{frame} %%Eine Folie
  \frametitle{Text für die SRA} %%Folientitel

  Aus den Vorgaben wurde der folgende Text formuliert und in die SRA eingebaut:
% Definitionsblock
  \begin{block}{S04.7: Allgemeine Anforderungen an die Website}
	  Geolocation soll festgelegt sein
  \end{block}
  
% Definitionsblock
  \begin{block}{S04.12: Allgemeine Anforderungen an die Website}
	  Auf der Startseite soll ein Video über einen Button gestartet werden
  \end{block}

% Definitionsblock
  \begin{block}{S04.15: Allgemeine Anforderungen an die Website}
	  Slideshow
  \end{block}
  	

\end{frame}


% SAD
\section{SAD}
% \subsection{Analysephase}
\begin{frame} %%Eine Folie
  \frametitle{Analyse} %%Folientitel

  Um einen besseren Überblick zu Erhalten wurden die folgenden Fragen gestellt:
  \bigskip
% Aufzählung
  \begin{enumerate}
   \item Welche Grundanforderungen sind nötig?
   \item Welche Sprachen werden benötigt?
  \end{enumerate}

\end{frame}
% \addtocounter{enumi}{4}

\begin{frame} %%Eine Folie
  \frametitle{Grundanforderungen} %%Folientitel

   Für die Registrierung sind folgende Grundmerkmale nötig:
   \bigskip
%   Auflsitung mit Punkt davor
   \begin{itemize}
    \item Daten des Nutzers müssen eingegeben werden können
    \item Die Nutzerdaten müssen sich übertragen lassen
    \item Die Nutzerdaten müssen vom System geprüft werden
   \end{itemize}

\end{frame}


\begin{frame} %%Eine Folie
  \frametitle{Sprachen} %%Folientitel

%   Geschachtelte Auflsitung mit Punkt davor
   \begin{itemize}
    \item Eingabe der Daten
    \begin{itemize}
     \item HTML5 für die Formulare
	 \item HTML5 zur Dateneingabe
    \end{itemize}

    \item Übertragung der Daten
    \begin{itemize}
     \item POST zum Übertragen
    \end{itemize}

    \item Überprüfung(Validierung der Daten)
    \begin{itemize}
     \item PHP zur Sprachauswahl
     \item JQuery-Validate zur Prüfung der Daten
    \end{itemize}

    \item Sonstiges
    \begin{itemize}
     \item CSS für das Design
    \end{itemize}

   \end{itemize}

\end{frame}

% Implementierung
\section{Implementierung}
\subsection{Code}

\defverbatim[colored]
  \makeset{
    \jscode

      \begin{lstlisting}[name=karte.js]
<script>		
	<!-- Karte vor Geolocation -->
	var gmegMap, gmegMarker, gmegInfoWindow, gmegLatLng;
	function gmegInitializeMap(){
	//Koordinaten werden auf Lat: 49.X / Lon: 12.X gesetzt (Hochschule)
	gmegLatLng = new google.maps.LatLng(49.004065,12.095247);
	//Karte mit den oben genannten Koordinaten wird erstellt 
	gmegMap = new google.maps.Map(document.getElementById("map"),{
		zoom:14,														//Zoomfaktor
		center:gmegLatLng,												//Koordinaten Hochschule
		mapTypeId:google.maps.MapTypeId.ROADMAP				
	});
	//Setzen des Markers
	gmegMarker = new google.maps.Marker({
		map:gmegMap,													//Marker in die oben erstellte Karte
		position:gmegLatLng											//Koordinaten Hochschule
	});
	//Erstellen eines Infofensters
	gmegInfoWindow = new google.maps.InfoWindow({
		content:'<b>Hochschule</b><br>Seybothstrasse<br>Regensburg'		
	});
	
	gmegInfoWindow.open(gmegMap,gmegMarker)
	;}
	
	//Listener
	google.maps.event.addDomListener(window,"load",gmegInitializeMap);
																				
</script>






      \end{lstlisting}
  }

\begin{frame}
  \frametitle{Standartkarte}
  \makeset
\end{frame}

\defverbatim[colored]
  \makeset{
    \jscode

      \begin{lstlisting}[name=karte.js]
      
	function create_route() {
		var directionsService = new google.maps.DirectionsService(),
			directionsDisplay = new google.maps.DirectionsRenderer(),
			createMap = function (start) {
				var travel = {
						origin : (start.coords)? new google.maps.LatLng(start.lat, start.lng) : start.address,
						destination : "Siemensstrasse 12, Regensburg",
						travelMode : google.maps.DirectionsTravelMode.DRIVING
						
					},
					mapOptions = {
						zoom: 10,
						//Default View:
						center : new google.maps.LatLng(49.0145423, 12.100855899999942),
						mapTypeId: google.maps.MapTypeId.ROADMAP
					};

				map = new google.maps.Map(document.getElementById("map"), mapOptions);
				directionsDisplay.setMap(map);
				
				//Callback abfragen
				directionsService.route(travel, function(result, status) {
					if (status === google.maps.DirectionsStatus.OK) {
						directionsDisplay.setDirections(result);
					}
				});
			};
			
			
			// Cookie noch nicht vorhanden
			if (!$.cookie("posLat")) {
				navigator.geolocation.getCurrentPosition(function (position) {
						
						// Positon erfolgreich gefunden!
						latitude = position.coords.latitude;
						longitude = position.coords.longitude;
						
						$.cookie("posLat", latitude);		//schreibt Position in Cookie
						$.cookie("posLon", longitude);
						
						createMap({							//erstellt die Map
							coords : true,
							lat : latitude,
							lng : longitude
						});
					},
					//Geolocation nicht verfuegbar
					alert('Ihr Standort konnte nicht bestimmt werden!')
				);
			}
			//Cookie vorhanden
			else {
				latitude = $.cookie("posLat");					//in posLat ist "latitude" gespeichert
				longitude = $.cookie("posLon");					//in posLon ist "longitude" gespeichert
				createMap({										//createMap siehe oben
					coords : true,
					lat : latitude,
					lng : longitude
				});
			}
	};
</script>






      \end{lstlisting}
  }

\begin{frame}
  \frametitle{Karte - Route}
  \makeset
\end{frame}

\defverbatim[colored]
  \makeset{
    \jscode

      \begin{lstlisting}[name=karte.js]
      
			// Cookie noch nicht vorhanden
			if (!$.cookie("posLat")) {
				navigator.geolocation.getCurrentPosition(function (position) {
						
						// Positon erfolgreich gefunden!
						latitude = position.coords.latitude;
						longitude = position.coords.longitude;
						
						$.cookie("posLat", latitude);		//schreibt Position in Cookie
						$.cookie("posLon", longitude);
						
						createMap({							//erstellt die Map
							coords : true,
							lat : latitude,
							lng : longitude
						});
					},
					//Geolocation nicht verfuegbar
					alert('Ihr Standort konnte nicht bestimmt werden!')
				);
			}
			


      \end{lstlisting}
  }

\begin{frame}
  \frametitle{Route, wenn Cookie nicht vorhanden}
  \makeset
\end{frame}

\defverbatim[colored]
  \makeset{
    \jscode

      \begin{lstlisting}[name=karte.js]
      
		
			//Cookie vorhanden
			else {
				latitude = $.cookie("posLat");					//in posLat ist "latitude" gespeichert
				longitude = $.cookie("posLon");					//in posLon ist "longitude" gespeichert
				createMap({										//createMap siehe oben
					coords : true,
					lat : latitude,
					lng : longitude
				});
			}
	};
</script>






      \end{lstlisting}
  }

\begin{frame}
  \frametitle{Route, wenn Cookie vorhanden}
  \makeset
\end{frame}

\defverbatim[colored]
  \makeset{
    \jscode

      \begin{lstlisting}[name=karte.js]
      
			// Cookie noch nicht vorhanden
			if (!$.cookie("posLat")) {
				navigator.geolocation.getCurrentPosition(function (position) {
						
						// Positon erfolgreich gefunden!
						latitude = position.coords.latitude;
						longitude = position.coords.longitude;
						
						$.cookie("posLat", latitude);		//schreibt Position in Cookie
						$.cookie("posLon", longitude);
						
						createMap({							//erstellt die Map
							coords : true,
							lat : latitude,
							lng : longitude
						});
					},
					//Geolocation nicht verfuegbar
					alert('Ihr Standort konnte nicht bestimmt werden!')
				);
			}
			


      \end{lstlisting}
  }

\begin{frame}
  \frametitle{Route, wenn Cookie nicht vorhanden}
  \makeset
\end{frame}

\defverbatim[colored]
  \makeset{
    \jscode

      \begin{lstlisting}[name=slide.js]
      
		
$(document).ready(function() { 	// Wenn document bereit ist, 
var slide=$('#show div');		// wird Variable erstellt, die auf den Div-Container verweist indem die Bilder gespeichert sind
	slide.fadeOut(10);			

fade(slide.first());			// Verweis auf erstes Bild

function fade ($ele) {
	$ele.fadeIn(300).delay(5000).fadeOut(300, function(){
		var $next = $(this).next();		// naechstes Bild
		
		// Sozusagen die Schleife. Wenn alle Bilder gezeigt wurden, gehe zum ersten
		fade($next.length>0 ? $next : $(this).parent().children().first()); 
	});
};

});


      \end{lstlisting}
  }

\begin{frame}
  \frametitle{Slideshow}
  \makeset
\end{frame}

\defverbatim[colored]
  \makeset{
    \jscode

      \begin{lstlisting}[name=video.js]
      
		
<script> var vid, playbtn, seekslider, curtimetext, durtimetext, mutebtn, volumeslider, fullscreenbtn; 
	function intializePlayer(){ 
	
		//Objekte 
		vid = document.getElementById("my_video"); 
		playbtn = document.getElementById("playpausebtn"); 
		seekslider = document.getElementById("seekslider"); 
		curtimetext = document.getElementById("curtimetext"); 
		durtimetext = document.getElementById("durtimetext"); 
		mutebtn = document.getElementById("mutebtn"); 
		volumeslider = document.getElementById("volumeslider"); 
		fullscreenbtn = document.getElementById("fullscreenbtn"); 
		//Listener
		playbtn.addEventListener("click",playPause,false); 
		seekslider.addEventListener("change",vidSeek,false); 
		vid.addEventListener("timeupdate",seektimeupdate,false); 
		mutebtn.addEventListener("click",vidmute,false); 
		volumeslider.addEventListener("change",setvolume,false); 
		fullscreenbtn.addEventListener("click",toggleFullScreen,false); 
	} 
	window.onload = intializePlayer; 


      \end{lstlisting}
  }

\begin{frame}
  \frametitle{Video Initialisierung}
  \makeset
\end{frame}

\defverbatim[colored]
  \makeset{
    \jscode

      \begin{lstlisting}[name=video.js]
      	
	// Zeitupdate
	function seektimeupdate(){ 
		var nt = vid.currentTime * (100 / vid.duration);
		seekslider.value = nt; 
		var curmins = Math.floor(vid.currentTime / 60); 
		var cursecs = Math.floor(vid.currentTime - curmins * 60); 
		var durmins = Math.floor(vid.duration / 60); 
		var dursecs = Math.floor(vid.duration - durmins * 60); 
		if(cursecs < 10){ cursecs = "0"+cursecs; } 
		if(dursecs < 10){ dursecs = "0"+dursecs; } 
		if(curmins < 10){ curmins = "0"+curmins; } 
		if(durmins < 10){ durmins = "0"+durmins; } 
		curtimetext.innerHTML = curmins+":"+cursecs; 
		durtimetext.innerHTML = durmins+":"+dursecs; 
	} 
	
	// Vollbild
		function toggleFullScreen(){
			if(vid.requestFullScreen){ 							// Mozilla Proposal
				vid.requestFullScreen(); 
			} else if(vid.webkitRequestFullScreen){ 	// Chrome / Safari
				vid.webkitRequestFullScreen(); 
			} else if(vid.mozRequestFullScreen){ 			// Firefox
				vid.mozRequestFullScreen(); 
			} 
		} 
	</script> 

      \end{lstlisting}
  }

\begin{frame}
  \frametitle{Video Initialisierung}
  \makeset
\end{frame}

\subsection{Demonstration}
\begin{frame} %%Eine Folie
%   \frametitle{Analyse} %%Folientitel

% Fettgedruckt
  \textbf{Es folgt eine Demonstration ...}
\end{frame}


\end{document}
