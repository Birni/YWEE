\section{Implementierung}
\subsection{Code}
\defverbatim[colored]
  \makeset{
    \phpcode

      \begin{lstlisting}[name=upload.php]
<?php
/* Passwortaenderung fuer den Privaten Ordner -- Daniel Tatzel */

/* create_htpasswd: Generiert eine .htpasswd Datei mit Benutzername und Passwort fuer den Login zum Privaten Ordner */
    $DIR = $_SERVER["DOCUMENT_ROOT"] . "/test_02/private/";
    if ( file_exists($DIR . '.htpasswd') )
        unlink($DIR.'.htpasswd');

    $passwd = crypt($_POST['passwd']);
    $htpasswd = 'admin:' . $passwd . "\n";

    $stream = "# Autor: Daniel Tatzel\n\n".$htpasswd;

    $handle = fopen($DIR . '.htpasswd',"a");
    fputs($handle,$stream);
    fclose($handle);

    header("Location: http://www.ebenezer-kunatse.net/private/");
?>
      \end{lstlisting}
}

\begin{frame}
  \frametitle{Passwort ändern}
  \makeset
\end{frame}

\defverbatim[colored]
  \makeset{
    \phpcode

      \begin{lstlisting}[name=delete.php]
<?php
    // Autor von delete.php: Daniel Tatzel
    // Loescht eine Datei

    $path = realpath('files/'.$_POST['file']);

    // Wenn vorhanden, dann l&oeschen
    if( is_file($path) )
        unlink( $path );
?>
      \end{lstlisting}
  }

\begin{frame}
  \frametitle{Löschen einer Datei}
  \makeset
\end{frame}

\defverbatim[colored]
  \makeset{
    \phpcode

      \begin{lstlisting}[name=ListDir.php]
<?php
    // Autor von ListDir.php: Daniel Tatzel
    // Liest alle Dateien aus, berechnet die Groesse in KB und erzeugt ein JSON objekt

    $Dir = "./files/";	// Zielordner

    // Kann dann mit JSON in Javascript umgewandelt werden zur Verarbeitung ( Array )
    $FileListing = array_diff(scandir($Dir), array('..', '.'));

    // Wird nur ausgefuehrt falls Daten vorhanden sind
    if ( count( $FileListing ) > 0 )
    {
        $file_arr = array();

        foreach ( $FileListing as $File )
        {
            $size = number_format( filesize($Dir.$File)/1024, 2, ",", "." );

            // Array fuer JSON Objekt
            $file_arr[] = array("name" => $File, "size" => $size);
        }

        echo json_encode( $file_arr );	// JSON Ausgabe
    }
?>
      \end{lstlisting}
  }

\begin{frame}
  \frametitle{Verzeichnis Inhalt holen}
  \makeset
\end{frame}

\defverbatim[colored]
  \makeset{
    \phpcode

      \begin{lstlisting}[name=upload.php]
<?php
    // Autor von upload.php: Daniel Tatzel
    // Laedt eine ausgewaehlte Datei auf den Server hoch

    $temp = explode(".", $_FILES["file"]["name"]);
    $extension = end($temp);

    // Gibt eine Meldung aus, falls ein Fehler aufgetreten ist
    if ($_FILES["file"]["error"] > 0)
    {
        echo "Fehler Code: " . $_FILES["file"]["error"] . "<br>";
    }
    else
    {
        // Prueft ob Datei schon vorhanden ist und gibt eine Meldung aus, ansonsten wird die Datei hochgeladen
        if (file_exists($_SERVER["DOCUMENT_ROOT"] . "/test_02/private/files/" . $_FILES["file"]["name"])) {
          echo $_FILES["file"]["name"] . " bereits vorhanden. ";
        }
        else {
          move_uploaded_file($_FILES["file"]["tmp_name"],
          $_SERVER["DOCUMENT_ROOT"] . "/test_02/private/files/" . $_FILES["file"]["name"]);
        }
    }
?>

      \end{lstlisting}
  }

\begin{frame}
  \frametitle{Datei hochladen}
  \makeset
\end{frame}

\defverbatim[colored]
  \makeset{
    \jscode

      \begin{lstlisting}[name=hide.js]
<script language="javascript">
    function show(selector)
    {
        var sel = selector;

        if ( sel == 0 )
            sel = "passwd";
        else
            sel = "file"

        if(document.getElementById( sel ).style.display == 'block')
        {
            document.getElementById( sel ).style.display = 'none';
        }
        else
        {
            document.getElementById( sel ).style.display = 'block';
        }
    }
</script>
      \end{lstlisting}
  }

\begin{frame}
  \frametitle{Ausblenden von Bereichen}
  \makeset
\end{frame}

\defverbatim[colored]
  \makeset{
    \jscode

      \begin{lstlisting}[name=table.js]
$(document).ready(function() {
    var mTable = $('#filesTable').dataTable({
        "oLanguage": {
            "sSearch": "Dateien filtern:",
            "oPaginate": {
                "sNext": "Naechste Seite",
                "sPrevious": "Vorherige Seite"
            },
            "sInfo": "Es wurden _TOTAL_ Dateien gefunden. Aktuelle Anzeige (_START_ bis _END_)",
            "sLengthMenu": "Zeige _MENU_ Dateien"
        }
    });
    mTable.fnClearTable();
    jQuery.each(files, function() {
        mTable.fnAddData(['<a href="files/' + this.name + '" > ' + this.name + '</a>', this.size, '<form method="POST" action=""><input type="hidden" name="file" value="'+ this.name +'"><input type="submit" name="delete" value="L&ouml;schen"></form>']);
    });
    mTable.draw();
  });
      \end{lstlisting}
  }

\begin{frame}
  \frametitle{Erstellen der Tabelle}
  \makeset
\end{frame}

\subsection{Demonstration}
\begin{frame} %%Eine Folie
  \frametitle{Demonstration} %%Folientitel

% Fettgedruckt
  \center
  \textbf{Es folgt eine Demonstration ...}
\end{frame}

\subsection{Ende}
\begin{frame}
  \frametitle{Ende}
  \center
 \textbf{\highlighton{Vielen Dank für Ihre Aufmerksamkeit}}
\end{frame}