\section{SAD}
\subsection{Analysephase}
\begin{frame} %%Eine Folie
  \frametitle{Analyse} %%Folientitel

  Um einen besseren Überblick zu Erhalten wurden die folgenden Fragen gestellt:
% Aufzählung
  \begin{enumerate}
   \item Welche Grundanforderungen sind nötig?
   \item Welcher Webserver steht zur Verfügung?
   \item Welche Möglichkeiten gibt es zur Umsetzung?
   \item Welche Sprachen werden benötigt?
  \end{enumerate}

\end{frame}
% \addtocounter{enumi}{4}

\begin{frame} %%Eine Folie
  \frametitle{Grundanforderungen} %%Folientitel

   Um den "`Privaten Ordner"' effektiv nutzen zu können sind folgende Grundmerkmale nötig:
%   Auflsitung mit Punkt davor
   \begin{itemize}
    \item Man muss das Passwort ändern können
    \item Man muss Dateien auswählen, hochladen und löschen können
    \item Die Dateien müssen Optisch angemessen aufgelistet werden
   \end{itemize}

\end{frame}

\begin{frame} %%Eine Folie
  \frametitle{Webserver} %%Folientitel
% Aufzählung
  \begin{enumerate}
   \item Webserver
   \begin{itemize}
    \item Auf dem Zielsystem ist ein \textbf{Apache} Webserver in der Version 2 vorinstalliert.
   \end{itemize}

   \item Bietet der Webserver Lösungen für das Problem?
   \begin{itemize}
    \item htaccess: Unterstützt die Zugriffskontrolle per Passwortschutz
   \end{itemize}

  \end{enumerate}


\end{frame}

\begin{frame} %%Eine Folie
  \frametitle{Umsetzung} %%Folientitel

  Zur Umsetzung stehen zwei Möglichkeiten zur Verfügung:
%   Auflsitung mit Punkt davor
  \begin{itemize}
   \item Verzeichnisschutz mittels htaccess Dateien
   \item Eigene Implementierung
  \end{itemize}

  \bigskip

  Da eine geeignete Implementierung vorhanden ist und der Aufwand für eine eigene zu hoch ist, wurde sich für die Umsetzung mittels htaccess Dateien entschieden.
\end{frame}

\begin{frame} %%Eine Folie
  \frametitle{Sprachen} %%Folientitel

%   Geschachtelte Auflsitung mit Punkt davor
   \begin{itemize}
    \item Passwort ändern
    \begin{itemize}
     \item HTML5 für die Formulare
     \item PHP zum aktualisieren des Passworts
    \end{itemize}

    \item Dateien auswählen, hochladen und löschen
    \begin{itemize}
     \item HTML5 zum Auswählen
     \item PHP zum Hochladen und zum Löschen
    \end{itemize}

    \item Darstellung der Dateien
    \begin{itemize}
     \item HTML5 für die Definition der Tabelle
     \item PHP zum Ermitteln der Dateien
     \item JQuery zur Generierung der Tabelle
    \end{itemize}

    \item Sonstiges
    \begin{itemize}
     \item Javascript für das Design
    \end{itemize}

   \end{itemize}

\end{frame}

\subsection{Die Formulierung}
\begin{frame} %%Eine Folie
  \frametitle{Text für die SAD} %%Folientitel

  Aus den Ergebnissen der Analysephase wurde der folgende Text formuliert und in die SAD eingebaut:

% Definitionsblock
  \begin{block}{}
	Blablablba muss ich noch formulieren
  \end{block}

\end{frame}