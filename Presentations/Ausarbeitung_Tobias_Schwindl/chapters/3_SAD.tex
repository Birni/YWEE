\section{SAD}
\subsection{Analysephase}
\begin{frame} %%Eine Folie
  \frametitle{Analyse} %%Folientitel

  Um einen besseren Überblick zu Erhalten wurden die folgenden Fragen gestellt:
% Aufzählung
  \begin{enumerate}
   \item Welche Grundanforderungen sind nötig?
   \item Welche Sprachen werden benötigt?
  \end{enumerate}

\end{frame}
% \addtocounter{enumi}{4}

\begin{frame} %%Eine Folie
  \frametitle{Grundanforderungen} %%Folientitel

   Um eine Nachrichtenseite erfolgreich zu implementieren, ist folgendes nötig:
%   Auflsitung mit Punkt davor
   \begin{itemize}
   	\item Alle User dürfen die Nachrichten lesen
    \item Besucher und eingeloggte User ohne Adminstatus dürfen das Formular nicht sehen.
    \item Der Admin muss Nachrichten schreiben können
   \end{itemize}

\end{frame}

\begin{frame} %%Eine Folie
  \frametitle{Sprachen} %%Folientitel

%   Geschachtelte Auflsitung mit Punkt davor
   \begin{itemize}
    \item Für das Formular
    \begin{itemize}
     \item HTML5
     \item JS um die Eingaben zu prüfen und die Nachrichten anzuzeigen
     \item JSON für das weitergeben der Daten an JS
    \end{itemize}

    \item Verhindern dass andere als der Amdin das Formular sehen
    \begin{itemize}
     \item PHP Adminabfrage im Code beim Aufrufen der Seite
    \end{itemize}

    \item Skript für das Speichern/Löschen der Nachrichten
    \begin{itemize}
     \item PHP für die SQL Queries 
    \end{itemize}
    

   \end{itemize}

\end{frame}

\subsection{Die Formulierung}
\begin{frame} %%Eine Folie
  \frametitle{Text für die SAD} %%Folientitel

  Aus den Ergebnissen der Analysephase wurde der folgende Text formuliert und in die SAD eingebaut:

% Definitionsblock
  \begin{block}{}
	Der Administrator hat die Möglichkeit auf der News Seite neue Nachrichten zu schreiben. Dazu gibt er den
	Betreff und die Nachricht in ein HTML 5 Formular ein. Diese werden in der Datenbank gespeichert und für alle
	sichtbar (auch für nicht angemeldete User). Der Admin hat außerdem die Möglichkeit per Adminverwaltung
	nicht mehr benötigte Nachrichten zu löschen.
  \end{block}

\end{frame}

\begin{frame} %%Eine Folie
  \frametitle{Spendenseite} %%Folientitel
	Für die Spendenseite ergibt sich ein ähnliches Bild, nur das alle angemeldeten User spenden können. 
  
\end{frame}

\begin{frame} %%Eine Folie
  \frametitle{Text für die SAD} %%Folientitel

  Aus den Ergebnissen der Analysephase wurde der folgende Text formuliert und in die SAD eingebaut:

% Definitionsblock
  \begin{block}{}
	Ein registrierter Benutzer kann per Kreditkarte Spenden. Dazu gibt er seine Kreditkarteninformationen in ein
	HTML5 Formular Kreditkartennummer, Ablaufdatum der Karte, die Prüfziffer und den zu zahlenden Betrag.
	Diese werden mit Javascript auf Richtigkeit der Informationen geprüft (HTML5 Formular) und dann für die
	weitere Verwendung in der MySQL Datenbank gespeichert.
  \end{block}

\end{frame}


